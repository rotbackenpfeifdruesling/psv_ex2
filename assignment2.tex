\documentclass[11pt,a4paper]{article}
\usepackage{latexsym}
\usepackage{paralist}
\usepackage{amssymb}
\usepackage{url}

\parskip        0mm
\oddsidemargin  0in
\evensidemargin 0in
\textwidth      6in
\topmargin      0in
\marginparwidth 30pt
\textheight     9.2in
\headheight     .1in
\headsep        .1mm

\title{VU Programm- und Systemverifikation\\
\Large Homework: Hoare Logic}
\author{Name: \underline{David Pernerstorfer}\qquad Matr. number: \underline{01633068}}
\date{{\bf Due date:} May 15, 2018, \underline{1pm}}

\usepackage{listings}
\lstset{
  basicstyle=\footnotesize\ttfamily, % Standardschrift
  numbers=left,               % Ort der Zeilennummern
  numberstyle=\tiny,          % Stil der Zeilennummern
  numbersep=5pt,              % Abstand der Nummern zum Text
  tabsize=2,                  % Groesse von Tabs
  extendedchars=true,         %
  breaklines=true,            % Zeilen werden Umgebrochen
  keywordstyle=\ttfamily,
  stringstyle=\ttfamily, % Farbe der String
  showspaces=false,           % Leerzeichen anzeigen ?
  showtabs=false,             % Tabs anzeigen ?
  xleftmargin=17pt,
  framexleftmargin=17pt,
  framexrightmargin=5pt,
  framexbottommargin=4pt,
  showstringspaces=false      % Leerzeichen in Strings anzeigen ?
}
\lstloadlanguages{
  C , C++, ml
}


\gdef\dash---{\thinspace---\hskip.16667em\relax}
\gdef\smdash--{\thinspace--\hskip.16667em\relax}
\newcommand{\emn}[1]{{\em #1\/}}
\gdef\op|{\,|\;}

\newcommand{\true}{{\sc true}}

\begin{document}

\maketitle

\paragraph{Task 1 (5 points):}
Prove the Hoare Triple below (assume that the domain of
all variables except {\tt r} in the program are the natural numbers
including 0, and {\tt r} is a Boolean variable). The operator
{\tt\%} is the modulo operator of the C programming language
(equivalent to {\sf mod} in the post-condition).
You need to find a sufficiently strong loop invariant.
%%
Annotate the following code directly with the required assertions.
Justify each assertion by stating which Hoare rule you used
to derive it.

\newpage

\begin{tabbing}
  \quad\=\quad\=\quad\=\\\kill
  $\{\mathsf{true}\}$\\[5ex]
  $\{\mathsf{true}\}$\\[2.5ex]
  {\tt r = false;}\\[2.5ex]
  $\{ (\neg r)\}$\\[2.5ex]
  {\tt i = n;}\\[2.5ex]
  $\{ ((\neg r) \, \vee \, ((i < n) \,\wedge \, (i > 2) \, \wedge \, (n~\mathsf{mod}~i = 0))) \}$\\[2.5ex]
  {\tt while ((i > 2) $\wedge$ (!r)) \{}\\[2.5ex]
  \>$\{ ((\neg r) \, \vee \, ((i < n) \,\wedge \, (i > 2) \, \wedge \, (n~\mathsf{mod}~i = 0))) \}$\\[2.5ex]
  \>{\tt i = i - 1;}\\[2.5ex]
  \>{\tt if (n \% i == 0)}\\[2.5ex]
  \>\>{\tt r = true;}\\[2.5ex]
  \>{\tt else }\\[2.5ex]
  \>\>{\tt skip;}\\[2.5ex]
  \>$\{ ((\neg r) \, \vee \, ((i < n) \,\wedge \, (i > 2) \, \wedge \, (n~\mathsf{mod}~i = 0))) \}$\\[2.5ex]
  {\tt \}}\\[2.5ex]
  % $\{ ((i \le 2) \, \vee \, r) \, \wedge \, ((\neg r) \, \vee \, ((i < n) \, \wedge \, (i > 2) \, \wedge \, (n~\mathsf{mod}~i = 0))) \}$\\[5ex]
  $\{ ((\neg r) \, \wedge \, (i \le 2)) \, \vee \, (r \, \wedge \, (i < n) \, \wedge \, (i > 2) \, \wedge \, (n~\mathsf{mod}~i = 0))) \}$\\[5ex]
  $\{ r\Rightarrow \left(\exists j\,.\,(j>1)\wedge(j<n)\wedge(n~\mathsf{mod}~j = 0)\right)\}$
\end{tabbing}

\clearpage

\paragraph{Task 2 (5 points):}

Prove the Hoare Triple below (assume that the domain of
all variables in the program are the integers,
i.e., ${\tt x}, {\tt y}, {\tt n}, {\tt m}\in\mathbb{Z}$.
You need to find a sufficiently strong loop invariant.
In the post-condition, $\vert n-m \vert$ denotes the absolute value of expression $n-m$.

Annotate the following code directly with the required assertions.
\underline{Justify} each assertion by stating which Hoare rule you used
to derive it, and the premise(s) of that rule. If you strengthen
or weaken conditions, explain your reasoning.

\begin{tabbing}
  \quad\=\quad\=\quad\=\\\kill
  $\{\mathsf{true}\}$\\[5ex]
  {\tt x := n;}\\[5ex]
  {\tt y := m;}\\[5ex]
  {\tt if (x > y) \{}\\[5ex]
  \>{\tt t := x;}\\[5ex]
  \>{\tt x := y;}\\[5ex]
  \>{\tt y := t;}\\[5ex]
  {\tt\} else \{ }\\[5ex]
  \>{\tt skip;}\\[5ex]
  {\tt \}}\\[5ex]
  {\tt while (x != 0)} \{\\[5ex]
  \>{\tt x := x - 1;}\\[5ex]
  \>{\tt y := y - 1;}\\[5ex]
  {\}}\\[5ex]
  $\{(y = \vert n - m\vert)\}$
\end{tabbing}

\clearpage

\paragraph{Task 3 (5 points):}
Download the latest version of the
C Bounded Model Checker (\textsc{Cbmc}) from
\url{http://www.cprover.org/cbmc/} and familiarize
yourself with the tool using the manual you can find on the
same web-page. Use {\sc Cbmc} to detect the heartbleed
bug (which we discussed in the lecture) in the simplified code below,
and explain how you used the tool to detect the bug:
\begin{itemize}
\item Which \emph{unwinding depth} was required?
\item Which command-line parameters did you have to specify?
\item Which property was violated (as reported by {\sc Cbmc})?
\item Provide a fix for the bug!
  \begin{itemize}
    \item Provide your solution by
      providing a short code snippet and indicate by referring to the
      line numbers where it should be inserted in the listing below.
    \item \emph{In addition}, upload your fixed implementation as
      {\tt heartbleed\_fixed.c} on TUWEL.
    \item Use {\sc Cbmc} to verify that your solution is correct!
    \end{itemize}
\end{itemize}

\noindent
\begin{lstlisting}
#include <string.h>
#include <stdlib.h>

typedef struct {
  unsigned char type;
  unsigned char data[42];
  unsigned int len;
} ssl_buffer;

typedef struct {
  ssl_buffer buffer;
} SSL;

/* function stubs - we don't need the implementation */
void RAND_pseudo_bytes (unsigned char*, unsigned int);
int ssl3_write_bytes (SSL*, unsigned, void*, unsigned);
unsigned int nondet_uint();
unsigned char nondet_uchar();

#define n2s(c,s)	((s=(((unsigned int)(c[0]))<< 8)| \
			    (((unsigned int)(c[1]))    )),c+=2)
#define s2n(s,c)	((c[0]=(unsigned char)(((s)>> 8)&0xff), \
			  c[1]=(unsigned char)(((s)    )&0xff)),c+=2)

#define TLS1_HB_REQUEST		1
#define TLS1_HB_RESPONSE	2
#define TLS1_RT_HEARTBEAT 24

int tls1_process_heartbeat(SSL *s) {
  unsigned char *p = s->buffer.data, *pl;

  unsigned char hbtype;
  unsigned int payload;
  unsigned int padding = 16;

  hbtype = s->buffer.type;
  n2s(p, payload);
  pl = p;

  if (hbtype == TLS1_HB_REQUEST) {
    unsigned char *buffer, *bp;
    int r;

    /* Allocate memory for the response, size is 1 bytes
     * message type, plus 2 bytes payload length, plus
     * payload, plus padding
     */
    buffer = malloc(1 + 2 + payload + padding);
    bp = buffer;

    /* Enter response type, length and copy payload */
    *bp++ = TLS1_HB_RESPONSE;
    s2n(payload, bp);
    memcpy(bp, pl, payload);
    bp += payload;
    /* Random padding */
    RAND_pseudo_bytes(bp, padding);

    r = ssl3_write_bytes(s, TLS1_RT_HEARTBEAT, buffer, 3 + payload + padding);

    if (r < 0)
      return r;
  }
  else if (hbtype == TLS1_HB_RESPONSE) {
    // ...
  }

  return 0;
}

int main() {
  SSL obj;

  obj.buffer.type = TLS1_HB_REQUEST;
  // non-deterministically assign a length of the buffer
  obj.buffer.len = nondet_uint();
  return tls1_process_heartbeat(&obj);
}
\end{lstlisting}

\vfill
The source code {\tt heartbleed.c} can also be downloaded from TISS.
\vfill

\clearpage

\paragraph{Task 4 (5 points):}
Use the \textsc{Klee} symbolic simulator (using the Docker image
from \url{klee.github.io} as explained in the lecture) to test
the following implementation of Euclid's algorithm:

\noindent
\begin{lstlisting}
unsigned gcd (unsigned x, unsigned y)
{
  unsigned m, k;
  if (x > y) {
    k = x;
    m = y;
  }
  else {
    k = y;
    m = x;
  }

  while (m != 0) {
    unsigned r = k % m;
    k = m; m = r;
  }
  return k;
}
\end{lstlisting}

Use \textsc{Klee} to generate test inputs from the following
\emph{specification}:

\noindent
\begin{lstlisting}
#define MIN(x, y) ((x)<(y))?(x):(y)
#define MAX(x, y) ((x)<(y))?(y):(x)
#define IS_CD(r, x, y) (((x)%(r)==0)&&((y)%(r)==0))

unsigned gcd (unsigned x, unsigned y)
{
  for (unsigned t = MIN (x,y); t>0; t--) {
    if (IS_CD(t, x, y))
      return t;
  }
  return MAX(x, y);
}
\end{lstlisting}

(The source code of both implementations can be downloaded from TISS.)

\begin{itemize}
\item How many test cases are required \emph{at least} to
  achieve branch coverage for the implementation?
\item Provide a \emph{minimal} number of test cases generated
  with {\sc Klee} such that branch coverage for the implementation
  is achieved!
  \begin{center}
    \begin{tabular}{|c|c||c|}
      \hline
      {\tt x} & {\tt y} & {\tt gcd(x,y)} \\
      \hline
      0 & 0 & 0 \\
      1 & 2 & 0 \\
      & &\\[7ex] \\
    \end{tabular}
  \end{center}
  \vfill
  \emph{In addition}, upload the {\tt ktest} files representing
  your test cases above (zipped in a folder {\tt klee-last} and
  named {\tt test00000$n$.ktest}) to TUWEL. Make sure the
  symbolic variables are called {\tt x} and {\tt y}!
\end{itemize}

\clearpage
\begin{itemize}
\item If a given test suite achieves branch coverage
  for the specification, does the same test suite also
  achieve branch coverage for the implementation \ldots
  \begin{compactitem}
  \item for this specific example?
  \item in general?
  \end{compactitem}
  For both cases, explain why!
\end{itemize}

{\bf Hint:} To replay the test cases, you need to make sure that
the environment variable {\tt LD\_LIBRARY\_PATH} points to the
directory {\tt /home/klee/klee\_build/klee/lib} containing
the library {\tt libkleeRuntest.so.1.0}!

\vfill
Upload a pdf file with your solutions to TUWEL by May 15, 2018.

\end{document}
